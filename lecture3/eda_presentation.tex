\documentclass{beamer}
\usetheme{Antibes}
\usepackage[noae]{Sweave}
\usepackage{multirow}
%\usepackage{verbatim} 

\title{Exploratory Data Analysis}
\author{Damian Thomas}
\date{2017-02-03}

\begin{document}

%===============================================================================


%---------------------------------------
\begin{frame}
\maketitle

\end{frame}


%---------------------------------------
\begin{frame}
\frametitle{Topics}

\tableofcontents

\end{frame}



%---------------------------------------
\section{Review}
\subsection{R Data Structures}
\begin{frame}
\frametitle{R Data Structures}

\begin{table}
\caption{R Data Structures \par by Content Type and Number of Dimensions}
\begin{tabular}{c|ll}
   & Homogeneous   & Heterogeneous \\ \hline
1d & Atomic vector & List \\
2d & Matrix        & Data Frame \\
$n$d & Array       & \\ 

\end{tabular}
\end{table}

\end{frame}



%---------------------------------------
\subsection{Brackets}
\begin{frame}
\end{frame}



%---------------------------------------
\subsection{Subsetting}
\begin{frame}
\end{frame}



%---------------------------------------
\subsection{Iteration}
\begin{frame}
\end{frame}



%---------------------------------------
\subsection{Functions}
\begin{frame}
\end{frame}



%---------------------------------------
\section{Importing Data}
\begin{frame}
\end{frame}



%---------------------------------------
\begin{frame}[fragile]
\frametitle{Anscombe's Quartet}

\begin{Schunk}
\begin{Sinput}
> # 
> setwd("~/projects/hu/Ecog314_Spring2017/lecture3/")
> anscombe <- read.csv("data/anscombe.csv")
> str(anscombe)
\end{Sinput}
\begin{Soutput}
'data.frame':	11 obs. of  8 variables:
 $ x1: int  10 8 13 9 11 14 6 4 12 7 ...
 $ x2: int  10 8 13 9 11 14 6 4 12 7 ...
 $ x3: int  10 8 13 9 11 14 6 4 12 7 ...
 $ x4: int  8 8 8 8 8 8 8 19 8 8 ...
 $ y1: num  8.04 6.95 7.58 8.81 8.33 ...
 $ y2: num  9.14 8.14 8.74 8.77 9.26 8.1 6.13 3.1 9.13 7.26 ...
 $ y3: num  7.46 6.77 12.74 7.11 7.81 ...
 $ y4: num  6.58 5.76 7.71 8.84 8.47 7.04 5.25 12.5 5.56 7.91 ...
\end{Soutput}
\end{Schunk}
\end{frame}



%---------------------------------------
\section{Exploratory Data Analysis}


\subsection{What is it?} 

\begin{frame}
\frametitle{What is Exploratory Data Analysis?} 

\begin{quote}
In statistics, exploratory data analysis (EDA) is an approach to analyzing data sets to summarize their main characteristics, often with visual methods.\footnote{Source: \url{https://en.Wikipedia.org/wiki/Exploratory_data_analysis}}
\end{quote}

\end{frame}



%---------------------------------------
\begin{frame}
\frametitle{EDA is:}

\begin{itemize}
\item Data focused
\item Informal. No model is specified
\item Gain insight into the data generating process. 
\item Learn about the data, underlying structure
\item Summarize the data without losing information. 
\item Gather key information required to build a model. 
\item Generate questions 
\item help decide what sort of model fits
\end{itemize}

\end{frame}



%---------------------------------------
\begin{frame}
\frametitle{EDA is \em{not}:}

\begin{itemize}
\item Model focused
\item Dependent on assumptions (randomness, normality, etc.)
\item A rigorous formal approach
\item Model Specification (regressions, ANOVA)
\item Parameter estimation
\item Hypothesis testing \slash statistical inference
\end{itemize}

\end{frame}





%---------------------------------------
\section{Techniques}
\begin{frame}

\frametitle{Techniques}
\begin{itemize}
\item Summary statistics
\item Visualizations
\end{itemize}

\end{frame}




%---------------------------------------
\subsection{Five number summary}
\begin{frame}

\frametitle{Tukey's five number summary}

\begin{itemize}
\item minimum: smallest value
\item lower quartile: 25th percentile
\item median: middle value
\item upper quartile: 75th percentile
\item maximum: largest value
\end{itemize}

\end{frame}



%---------------------------------------
\begin{frame}[fragile]
\frametitle{Tukey's five number summary in R}

\end{frame}



%---------------------------------------
\begin{frame}

\end{frame}


%---------------------------------------
\subsection{Five number summary}
\begin{frame}

Key summary statistics 
\begin{itemize}
\item Extremes
\item Location
\item Spread
\end{itemize}

\end{frame}

%---------------------------------------
\subsection{Box plot}
\begin{frame}
\end{frame}

%---------------------------------------
\section{}
\begin{frame}
\frametitle{}

\begin{itemize}
\item
\item
\item
\end{itemize}


\end{frame}


%===============================================================================
\end{document}


\begin{itemize}
\item Location
\item Spread
\item Extreme values
\item Shape
\end{itemize}

\begin{comment}
\item Visualizations
    \begin{itemize}
    \item Box plot
    \item Scatter plot
    \item Line plot
    \item Bar plot
    \item Histogram
    \end{itemize}
\end{itemize}
\end{comment}

%---------------------------------------
\section{}
\begin{frame}
\frametitle{}

\begin{itemize}
\item
\item
\item
\end{itemize}

\end{frame}
